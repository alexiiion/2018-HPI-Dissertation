% the abstract

% Recently, researchers started to engineer not only the outer shape of objects, but also their internal microstructure. Such objects, typically based on 3D cell grids, are also known as metamaterials. Metamaterials have been used, for example, to create materials with soft and hard regions. 

% So far, metamaterials were understood as materials---we want to think of them as machines. In this dissertation, we push the area of metamaterials further by integrating mechanical functions as well as computational abilities into the material structure. 

% We demonstrate metamaterial objects that behave like analog machines, i.e., they perform a \textit{mechanical function}. Such metamaterial mechanisms consist of a single block of material the cells of which play together in a well-defined way in order to achieve macroscopic movement. Our metamaterial door latch, for example, transforms the rotary movement of its handle into a linear motion of the latch. Our metamaterial Jansen walker consists of a single block of cells---that can walk. The key element behind our metamaterial mechanisms is a specialized type of cell, the only ability of which is to shear. 

% While our approach offers tangible benefits for users (e.g., it solves mechanical problems in a single part, thereby eliminates the need for assembly), we see the main promise of this work in that it allows us to achieve a deeper integration between the structural and the mechanical functions of materials. 

% We continue on this path by investigating how to \textit{integrate logical functions} into materials. We explore how to embody mechanical computation into 3D printed objects, i.e., without electronic sensors, actuators, or controllers typically used for this purpose. A key benefit of our approach is that the resulting objects can be 3D printed in one piece and thus do not require assembly. We are building on 3D printed cell structures, also known as metamaterials. We introduce a new type of cell that propagates a digital mechanical signal using an embedded bistable spring. When triggered, the embedded spring discharges and the resulting impulse triggers one or more neighboring cells, resulting in signal propagation. We extend this basic mechanism to implement simple logic functions. We demonstrate interactive objects based on this concept, such as a combination lock. 

% Pushing metamaterials even further, we propose changing material properties \textit{afer fabrication}. The key idea is to provide a technique that allows fabricating 3D objects fast by, e.g., injection molding them rather than 3D printing, which is slow. Users customize the motion path of mass-fabricated metamaterial mechanisms by switching cells from being able to shear to rigid. Thereby, we can fabricate many identical copies of a, e.g., metamaterial Jansen walker and users can change the walking motion after it was fabricated. The cell that enables these state changes is the snapping cell---it can change its state by incorporating a snap-fit connector. The same approach allows us to demonstrate a mass-fabricated insole, which users customize at home to fit their feet. To achieve this, they switch selected cells from their rigid state into their soft state, i.e., they change the material's compliance in selected areas. 

% Designing cell structures that allow for such behaviour are hard to design: (1) it is non-obvious how a given structure will deform, and (2) inventing a new cell structure based on a desired deformation is a lengthy iterative process. To allow users to benefit in the future from such complex cell structures, we implemenet a specialized 3D editor. It allows users to model new metamaterial cells and to create entire 3D shapes from their structures. To help users verify their novel cell designs during editing, our editor allows users to apply forces and simulates how the object deforms in response. Our editor allows users to route signals, simulate signal flow, and synthesize cell patterns. Furthermore, we support users by integrating expert knowledge into the editor, e.g.,  we highlight parts that prevent previously moving parts from deforming in real-time during the design process. 


% We conclude by ...


% DAAD abstract

Digital fabrication machines such as 3D printers excel at producing arbitrary shapes, such as for decorative objects. In recent years, researchers started to engineer not only the outer shape of objects, but also their internal microstructure. Such objects, typically based on 3D cell grids, are known as metamaterials. Metamaterials have been used to create materials that, e.g., change their volume, or have variable compliance. 

While metamaterials were initially understood as materials, we propose to think of them as \textit{devices.}

We argue that thinking of metamaterials as devices enables us to create internal structures that offer functionalities to implement an \textit{input-process-output model} without electronics, but purely within the material’s internal structure. 
In this thesis, we investigate three aspects of such metamaterial devices that implement parts of the input-process-output model: 
(1)~materials that process analog inputs by implementing mechanisms based on their microstructure, 
(2)~that process digital signals by embedding mechanical computation into the object’s microstructure, and 
(3)~interactive metamaterial objects that output to the user by changing their outside to interact with their environment. 
The input to our metamaterial devices is provided directly by the users interacting with the device by means of physically pushing the metamaterial, e.g., turning a handle, pushing a button, etc. 

The design of such intricate microstructures, which enable the functionality of metamaterial devices, is not obvious. The complexity of the design arises from the fact that not only a suitable cell geometry is necessary, but that additionally cells need to play together in a well-defined way. 
% The microstructure of 3D printed metamaterial devices, which defines their processing and output capabilities, is equivalent to the object’s \textit{geometry.} 
To support users in creating such microstructures, we research and implement interactive design tools. 
These tools allow experts to freely edit their materials, while supporting novice users by auto-generating cells assemblies from high-level input. 
% These tools provide a library of structures that we found to be essential yet allow users to freely create their own structures. 
% We make these tools appeal to a broad audience by providing easy access via a web browser, by implementing easy-to-use interactions like brushing, by automating parts of the cells generation, by interactively simulating the cell structures’ deformation directly in the editor, and by providing easy export of the materials to the 3D printer. 
Our tools implement easy-to-use interactions like brushing, interactively simulate the cell structures’ deformation directly in the editor, and export the geometry as a 3D-printable file. 
Our goal is to foster more research and innovation on metamaterial devices by allowing the broader public to contribute.

% end DAAD abstract




% <Next, we want to persist the state of cells to change the material properties. ...>
% Integrating new functions into materials is one way to create a new rendition of 'smart materials'. 

% Metamaterials are as printed, we create a material that contains cells the state of which can be changed after fabrication in order to alter the material properties of an already printed object.

% In order to allow users to create metamaterial mechanisms efficiently we implemented a specialized 3D editor. It allows users to place different types of cells, including the shear cell, thereby allowing users to add mechanical functionality to their objects. To help users verify their designs during editing, our editor allows users to apply forces and simulates how the object deforms in response. 

% We present a custom editor that allows users to model 3D objects, route signals, simulate signal flow, and synthesize cell patterns. 

