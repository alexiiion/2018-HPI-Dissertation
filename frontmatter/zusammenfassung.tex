%!TEX root = ../dissertation.tex
% zusammenfassung

Digitale Fabrikationsmaschinen, wie 3D-Drucker, eignen sich hervorragend um beliebige Formen zu produzieren. 
Daher sind sie bei Endnutzern für die Erstellung von dekorativen Elementen sehr beliebt. 
Forscher hingegen haben in den letzten Jahren damit begonnen, nicht nur die äußere Form zu betrachten, sondern auch Mikrostrukturen im Inneren. 
Solche Strukturen, die meist auf einem 3-dimensionalen Gitter angeordnet sind, sind als "Metamaterialien" bekannt.
Metamaterialien wurden entwickelt, um Eigenschaften wie Volumenänderung oder lokalisiert die Steifheit des Materials zu steuern.

Traditionell werden Metamaterialien als Materialien betrachtet, wir hingegen betrachten sie als \textit{Geräte}.

In dieser Arbeit zeigen wir, dass die Betrachtung von Metamaterialien als Geräte es erlaubt Strukturen zu kreieren, die Geräte nach dem \textit{Eingabe-Verarbeitung-Ausgabe} Prinzip realisieren -- und das gänzlich ohne Elektronik.
Wir untersuchen 3 Aspekte von solchen funktionsfähigen Meta\-material-Geräten die jeweils Teile des EVA Prinzips implementieren: 
(1)~Materialien, die analoge Eingabe als Mechanismen die durch ihre Mikrostruktur bestimmt sind verarbeiten, 
(2)~Materialien, die digitale Eingabe verarbeiten und mechanische Berechnungen in ihrer Mikrostruktur durchführen und 
(3)~Materialien, die ihre äußere Textur dynamisch verändern können um mit dem Nutzer zu kommunizieren.
Die Eingabe für Meta\-ma\-te\-rial-Geräte ist in dieser Arbeit direkt durch den Nutzer gegeben, der mit dem Gerät interagiert, zum Beispiel durch Drücken eines Griffs, eines Knopfes, etc.

Das Design von solchen filigranen Mikrostrukturen, die die Funktionalität der Metamaterial-Geräte definiert, ist nicht offensichtlich oder einfach.
Der Designprozess ist komplex, weil nicht nur eine Zellstruktur gefunden werden muss, die die gewünschte Deformation durchführt, sondern die Zellstrukturen zusätzlich auf eine wohldefinierte Weise zusammenspielen müssen. 
Um Nutzern die Erstellung von diesen Mikrostrukturen zu ermöglichen, unterstützen wir sie durch interaktive Computerprogramme, die wir in dieser Arbeit untersuchen und implementieren. 
Wir haben Software entwickelt, die es Experten erlaubt die Mikrostrukturen frei zu platzieren und zu editieren, während Laien durch automatisch generierte Strukturen geholfen wird. 
Unsere Software beinhaltet einfach zu bedienende Interaktionskonzepte, wie zum Beispiel das aufmalen von funktionalen Eigenschaften auf Objekte, eine integrierte Vorschau der Deformation, oder der 3D-druckfähige Export der erstellten Geometrie. 
Das Ziel dieser Arbeit ist es langfristig Forschung und Innovation von Metamaterial-Geräten zu fördern, so dass sich sogar die breite Masse in das Thema einbringen kann. 
