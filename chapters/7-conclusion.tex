\chapter{Conclusion}
\label{chapter:conclusion}

While creating devices and mechanisms so far remained a privilege of experts only, in this thesis, we provide concepts, mechanical cell designs and interactive editing tools that enable everyone to design and fabricate new devices. 

Creating new devices typically consists of two phases, (1) the design phase and (2) the craft phase. The design phase involves the design of individual parts and their positioning with respect to each other in order to fulfil a function. The craft phase is concerned with the manufacturing and assembly of those parts. However, both phases usually require substantial knowledge and expertise with regards to fit, play, friction, etc. 

The main contribution of this thesis is that it eases both phases in order to enable experts and novices to create new devices. Our metamaterial devices require no assembly and the manufacturing is offloaded to the 3D printer, which shields users from the craft phase. We support users in the design phase by providing interactive computational tools that help creating the geometry of the cell structures, which define the function of the device.

On a more abstract level, we believe that allowing even novice users to participate in the innovation of new devices will accelerate technological advancement. Potentially billions of people could invent devices and help push the boundaries of technology as compared to only a few experts and researchers. 


\section{Contributions}

We extend the research fields of metamaterials, digital fabrication, and geometry processing by contributing a general-purpose approach to creating interactive and expressive metamaterial devices. Such metamaterial structures are a new genre that is of higher complexity and that exploits more degrees of freedom than previous work in this field, and that allow metamaterials to tackle problems they have traditionally not been able to address. 

\todo{add: we contribute cell design, and software, and deeper understanding, ...}

\section{Benefits \& limitations}

We extend the research fields of metamaterials, digital fabrication, and geometry processing by contributing a general-purpose approach to creating interactive and expressive metamaterial devices. Such metamaterial structures are a new genre that is of higher complexity and that exploits more degrees of freedom than previous work in this field, and that allow metamaterials to tackle problems they have traditionally not been able to address. 

Compared to traditional multi-part mechanisms, metamaterial devices offer several benefits. (1) The resulting devices consist of a single part. They can thus be created using particularly simple fabrication processes, such as single-material 3D printers (e.g., FDM printers). (2) As they consist of a single piece, they require no assembly. (3) Since the movement is performed by deformation there is virtually no friction, no need for lubrication, and thus for maintenance.

Since the functionality of our metamaterial devices is only defined by the deformation inside the cell structures, the resulting mechanical function is not dependent on engineering fit and play between parts to ensure a smooth motion. On the contrary, our metamaterials offer a direct relationship between the geometry and material. This relationship can be computed by software tools to enable novice users to develop new mechanisms and devices---a privilege that is so far reserved for experts.

However, our metamaterial devices are also subject to limitations. Since metamaterial mechanisms only rely on deformation, they are unable to produce continuous rotation. If continuous rotation is required, metamaterial mechanisms can be complemented with a separate axle. Metamaterial Textures are intended to be used on objects where the exact shape is not a critical property, since transitioning between textures relies on compression. And lastly, since our digital mechanical metamaterials implement combinational circuits, they consequently lack loops, a clock, and memory, and are therefore limited to much simple computing devices.

\section{Outlook}

\todo{add input-process-output diagram and illustrate ideas \\
- for sensing (input), \\
- for digital cells that create output textures to connect these two concepts \\
% - for connecting output to mechanisms, that output could then be sensed again creating a loop.
}

% \subsection{Designing materials that adapt to the human body}
% <jaws>